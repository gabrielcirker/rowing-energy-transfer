
\documentclass[11pt]{article}
\usepackage[margin=1in]{geometry}
\usepackage{amsmath, amssymb}
\usepackage{siunitx}
\usepackage{graphicx}
\usepackage{booktabs}
\usepackage{hyperref}
\hypersetup{colorlinks=true, linkcolor=black, citecolor=black, urlcolor=blue}

\title{Simulating Mechanical Energy Transfer in Rowing\\[0.5em]\large A Simplified Fluid--Dynamics and Mechanics Model}
\author{Gabriel Cirker}
\date{\today}

\begin{document}
\maketitle

\begin{abstract}
Rowing shells are propelled by intermittent strokes of multiple rowers. During
the drive phase each rower applies a force through the oar blade, generating thrust
on the boat, whereas during the recovery the boat coasts and decelerates
under hydrodynamic resistance. Understanding the conversion of rower effort
into boat motion requires a coupled description of the stroke cycle and the
drag force experienced by the hull. This study formulates a simple one--dimensional
model of a racing shell that balances a prescribed thrust profile against a
quadratic drag law. Newton's second law is integrated numerically over
multiple strokes to evaluate boat speed, energy input, drag losses and overall
efficiency. The model reproduces realistic speeds and demonstrates that
roughly six percent of the mechanical work delivered by the rowers appears as an
increase in kinetic energy, the remainder being dissipated by drag. The
framework illustrates how modest parameter changes influence performance and
provides an accessible platform for exploring rowing dynamics.
\end{abstract}

\section{Introduction}

Rowing is both a team sport and a mechanical system: multiple athletes work
together to convert metabolic energy into forward motion of a lightweight hull.
The oar blades interact with water and air, imparting momentum to the fluid
and in turn providing thrust to the boat. Meanwhile the hull experiences
hydrodynamic drag that opposes motion. For an eight--person shell with a
coxswain (an ``8+''), typical component masses are about \SI{650}{\kilogram}
for the rowers, \SI{90}{\kilogram} for the shell and \SI{25}{\kilogram} for the
oars, giving a total mass near \SI{765}{\kilogram}.

Rowing performance has steadily improved over time with advancements in
training and technology. Men's eights can cover the standard \SI{2000}{\metre}
race in about 5--6 minutes, corresponding to an average speed above
\SI{6}{\metre\per\second}. Achieving such speeds requires each rower to produce
on the order of \SI{400}{\watt} to \SI{500}{\watt} of power on average. Because
drag increases rapidly with speed, overcoming fluid resistance consumes the
vast majority of the rowers' energy.

This paper develops a minimalist time--domain model of mechanical energy
transfer in rowing. By prescribing a periodic thrust and balancing it
against quadratic drag, the model predicts speed oscillations over the stroke
cycle, quantifies energy losses and estimates the efficiency of the stroke.

\section{Mathematical Model}

\subsection{Equations of motion}

An 8+ shell of total mass $m$ subjected to a forward thrust
$F_{\text{thrust}}(t)$ and a resistive drag $F_d(v)$ obeys Newton's second law
in one dimension:
\begin{equation}
  m\,\frac{\mathrm{d}v}{\mathrm{d}t} = F_{\text{thrust}}(t) - F_d(v),
  \label{eq:newton}
\end{equation}
where $v(t)$ is the boat's velocity. The drag force is approximated by the
quadratic law
\begin{equation}
  F_d(v) = k_\mathrm{d} \, v^2,
  \label{eq:drag}
\end{equation}
where $k_\mathrm{d}$ is an effective drag coefficient (combining water and air
resistance).

\subsection{Thrust model}

Rowers deliver power in pulses. During the drive phase of each stroke the
oar blades engage the water and the crew pushes hard against the footplates,
producing thrust. During the recovery the blades are out of the water and no
propulsive force is applied. We model this as a periodic square--wave thrust:
\begin{equation}
  F_{\text{thrust}}(t) =
  \begin{cases}
    F_{\text{drive}}, & 0 \le \phi < \delta,\\[4pt]
    0, & \delta \le \phi < 1,
  \end{cases}
\end{equation}
where $\phi = (t \bmod T)/T$ is the phase over each stroke cycle of period $T$
(with $T=\SI{2.0}{\second}$ for a 30~spm rate). The parameter $0<\delta<1$ is
the duty fraction of the stroke spent in the drive; here we use $\delta=0.45$
(corresponding to a drive of $0.9$~s). The peak force $F_{\text{drive}}$ produced
by the crew during the drive is taken as \SI{3000}{\newton}, which is roughly
\SI{375}{\newton} per rower.

\subsection{Energy bookkeeping}

The mechanical work done by the rowers over a time interval is
\begin{equation}
  E_{\text{in}} = \int F_{\text{thrust}}(t)\,v(t)\,\mathrm{d}t.
\end{equation}
The energy dissipated by drag is
\begin{equation}
  E_{\text{drag}} = \int F_d(v)\,v(t)\,\mathrm{d}t = \int k_\mathrm{d} v(t)^3 \,\mathrm{d}t.
\end{equation}
The change in kinetic energy of the system is
\begin{equation}
  \Delta E_k = \tfrac{1}{2} m \left[v(t_{\text{final}})^2 - v(0)^2\right].
\end{equation}
A simple efficiency metric is then
\begin{equation}
  \eta = \frac{\Delta E_k}{E_{\text{in}}},
\end{equation}
representing the fraction of rower work that appears as kinetic energy of
the boat and crew. In steady racing conditions $\Delta E_k$ over many strokes
is small, so $\eta$ over a short interval is a better reflection of how
effectively added work increases speed.

\section{Numerical Method}

Equation~\eqref{eq:newton} is integrated numerically on a uniform time grid
with step $\Delta t = \SI{0.01}{\second}$. At each step the thrust is evaluated
based on the stroke phase and the velocity is updated via an explicit Euler
scheme. The simulation is initialized from rest ($v=0$) at the start of a
drive. The integration proceeds over multiple strokes until a desired total
simulation time is reached (typically \SI{20}{\second} for the cases
presented here). As the system approaches a limit cycle, we measure the
average velocity over the latter half of the piece to characterize the steady
speed.

The output of the simulation includes the time--series of velocity $v(t)$ and
thrust $F_{\text{thrust}}(t)$, as well as cumulative energy quantities. The
mechanical work done by the rowers, the energy dissipated by drag and the
change in kinetic energy are all computed by numerical integration.

\section{Results}

\subsection{Speed and thrust profiles}

Starting from rest, the model shell accelerates rapidly during the first few
strokes and reaches a repeating limit cycle of speed fluctuations by about
\SI{10}{\second}. Figure~\ref{fig:speed} shows the boat velocity $v(t)$ over a
\SI{20}{\second} simulation. The characteristic oscillatory pattern is
evident: during each drive phase the velocity rises quickly, then peaks near
the end of the drive, and falls during the recovery as drag decelerates the
hull. In the baseline scenario, the crew's power input eventually balances
the drag losses, yielding an approximately steady average speed of about
\SI{5.8}{\metre\per\second}. The maximum instantaneous speed reaches about
\SI{6.8}{\metre\per\second} at the end of each drive.

Figure~\ref{fig:thrust} depicts the applied thrust as a function of time.
By construction, the thrust alternates between \SI{3000}{\newton} during each
\SI{0.9}{\second} drive and zero during the recovery. These idealized square
pulses produce the sawtooth velocity curve of Figure~\ref{fig:speed}.

Over the \SI{20}{\second} simulation, the total energy input by the rowers is
approximately $1.4\times10^5$~J. Most of this energy is dissipated by drag,
while the remainder goes into increasing the kinetic energy of the system.
The efficiency $\eta$ is about 6--7\%, meaning only a small fraction of
the rowers' work appears as accelerated mass energy of the boat and crew.

\begin{figure}[t]
  \centering
  \includegraphics[width=0.8\textwidth]{speed_plot.png}
  \caption{Boat velocity as a function of time for the baseline simulation.
  The shell accelerates during each drive and decelerates during each
  recovery, approaching an oscillatory steady state.}
  \label{fig:speed}
\end{figure}

\begin{figure}[t]
  \centering
  \includegraphics[width=0.8\textwidth]{thrust_plot.png}
  \caption{Thrust force applied by the crew as a function of time in the
  model. During each drive interval a constant thrust of \SI{3000}{\newton}
  is applied, followed by a recovery with no propulsive force.}
  \label{fig:thrust}
\end{figure}

\subsection{Parametric variations}

The baseline simulation corresponds to an 8+ with standard parameters.
We explored how performance changes when varying key parameters like drag,
mass and stroke rate. Reducing the drag coefficient allows the shell to
attain higher steady speeds under the same power input. Conversely,
increasing drag reduces the top speed and efficiency (see Figure~\ref{fig:drag}).

Varying the total mass has little effect on the ultimate cruise speed (which
is set by the balance of power and drag), but it does affect acceleration.
A lighter boat accelerates more quickly, while a heavier crew or shell
accelerates more slowly (Figure~\ref{fig:mass}). Stroke rate also influences
performance: a higher stroke rate with the same force per stroke yields a
higher average speed, albeit at the cost of expending more energy per unit
time (Figure~\ref{fig:stroke}).

\begin{figure}[t]
  \centering
  \includegraphics[width=0.8\textwidth]{drag_effect.png}
  \caption{Effect of drag on velocity: lowering the effective drag coefficient
  leads to higher equilibrium speeds for the same rowing power.}
  \label{fig:drag}
\end{figure}

\begin{figure}[t]
  \centering
  \includegraphics[width=0.8\textwidth]{mass_effect.png}
  \caption{Effect of mass on velocity: decreasing the total mass produces
  faster acceleration and slightly higher peak speeds, though the steady
  cruise speed is dominated by drag and power.}
  \label{fig:mass}
\end{figure}

\begin{figure}[t]
  \centering
  \includegraphics[width=0.8\textwidth]{stroke_effect.png}
  \caption{Effect of stroke rate on velocity: increasing stroke rate
  (shorter cycle time) with the same force per stroke results in a higher
  average boat speed.}
  \label{fig:stroke}
\end{figure}

\section{Discussion}

The simplified model captures key features of rowing dynamics: oscillatory
speed, rapid initial acceleration and a limit cycle speed reached after
several strokes. By construction, the model cannot represent nuances like the
synchronization among rowers, flexibility of the boat, or variation in
individual rower output. Nonetheless, the results align reasonably well with
observations of real rowing. The predicted steady speed for an 8+ under these
conditions is on the same order as race-pace velocities, and the magnitude
of the speed fluctuations is similar to that seen in practice.

The model predicts that only a small fraction of the rowers' mechanical work
appears as kinetic energy of the system, with the rest lost to drag. This
aligns with the intuitive understanding that most of the energy in rowing is
used to overcome water resistance. The inefficiencies in the stroke further
emphasize the importance of technique in minimizing energy waste. Our
simplified model cannot capture all such effects, but it does illustrate the
magnitude of drag losses.

In future work, the model could be refined by including a time‑varying thrust
profile that more closely mimics real force application, as well as the
dynamic movement of rowers' mass. Nonetheless, even this minimal model
provides useful insight: it quantifies how changes in physical parameters
translate to changes in boat speed and efficiency.

\section{Conclusion}

We presented a simple physics‑based simulation of rowing, treating the boat
and crew as a single mass propelled by periodic thrust pulses against a
quadratic drag. Despite its simplicity, the model produces realistic boat
speed profiles and offers quantitative estimates of energy usage and
efficiency. According to the simulation, on the order of a few percent of
the rowers' output ends up as kinetic energy gain while the rest is
dissipated in overcoming drag. The model also illustrates how changes in
drag, crew mass or stroke rate can impact performance. These findings
reinforce the understanding that drag minimization and effective power
application are critical for rowing speed.

\bibliographystyle{plain}
\begin{thebibliography}{9}\itemsep -1pt
\bibitem{pulman}
C.\ Pulman.
\newblock {The Physics of Rowing}.
\newblock Technical report, University of Cambridge, 2007.

\bibitem{lazauskas}
L.\ Lazauskas.
\newblock {Rowing shell drag and hydrodynamics}.
\newblock Technical report, University of Adelaide, 1997.

\bibitem{kleshnev}
V.\ Kleshnev.
\newblock {Rowing biomechanics: Principles and applications}.
\newblock Rowing Biomechanics Newsletter, various issues.

\end{thebibliography}

\end{document}
